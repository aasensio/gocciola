\documentclass[12pt]{article}
\usepackage{natbib}
\usepackage[hypertex]{hyperref}


 \oddsidemargin 0.0in
 \evensidemargin 0.0in
 \textwidth 6.5in
 \topmargin -0.75in
 \textheight 9.5in
 \pagestyle{empty}

%%%%%%%%%%%%%%%%%%%%%%%%%%%%%%%%%%%%%%%%%%%%%%%%%%%%%%%%%%%%%%%%%%%

\def\separation {0.5cm}
\def\non{\nonumber \\}
\def\DnuD     {\hbox{$\Delta\nu_D$}}
\def\Jbar     {\hbox{$\bar J$}}
\def\j        {\hbox{$\jmath$}}
\def\N        {\hbox{$\cal N$}}
\def\Ie       {\hbox{$I_e$}}
\def\Ji       {\hbox{$\bar J^i_\mathrm{ext}$}}
\def\about    {\hbox{$\sim$}}
\def\x        {\hbox{$\times$}}
\def\half     {\hbox{$1\over2$}}
\def\Ncr      {\hbox{$N'_{\rm cr}$}}
\def\mic      {\hbox{$\mu$m}}
\def\ion#1#2  {#1\,{\small {#2}} }
\def\tot      {\tau_t}
\def\t(#1){\tau^{#1}}
\def\a(#1){\alpha^{#1}}
\def\G{Gocciola}
\def\M{Gocciola}
\def\LVG      {\texttt{LVG}}
\def\slab     {\texttt{slab}}

%%%%%%%%%%%%%%%%%%%%%%%%%%%%%%%%%%%%%%%%%%%%%%%%%%%%%%%%%%%%%%%%%%%


% For generation of the HTML manual with tth:
\def\tthdump#1{#1}      % For generating TeX source; ignored by tth
% Redefine symbols problematic for the browser:
%%tth:\def\ga{\hbox{$>\sim$}}
%%tth:\def\la{\hbox{$<\sim$}}
%%tth:\def\Mo{\hbox{$M_o$}}
%%tth:\def\Lo{\hbox{$L_o$}}
%%tth:\def\Mdot{\hbox{$M^{dot}$}}
%%tth:\def\Ivezic{Ivezic}

%%tth:\begin{html}<TITLE>User Manual for MOLPOP-CEP</TITLE>\end{html}
%%tth: This HTML file was generated from the TeX source by
%%tth: the translator TTH, which uses symbol fonts.  These fonts are
%%tth: not normally enabled for Netscape running under X, because of
%%tth: the way Netscape groups its fonts. If your browser has problems
%%tth: displaying the math symbols in this manual, an easy fix can be found
%%tth: on the TTH website at
%%tth:\begin{html}<A HREF="http://hutchinson.belmont.ma.us/tth/Xfonts.html">http://hutchinson.belmont.ma.us/tth/Xfonts.html</A>\end{html}
%%tth:\begin{html}<HR>\end{html}
%%%%%%%%%%%%%%%%%%%%%%%%%%%%%%%%%%%%%%%%%%%%%%%%%%%%%%%%%%%%%%%%%%%

\begin{document}

\title                  {\sc User Manual for \G}

\author{ A. Asensio Ramos \& J. Trujillo Bueno\\
         Instituto de Astrof\'{\i}sica de Canarias\\
         38205, La Laguna, Tenerife, Spain\\ \\
        \\[0.5in] \today}% \today}
\date{}
\maketitle


\tableofcontents

\newpage

\section*{Disclaimer}

This software is distributed ``as is'' and the authors do not take any
responsibility for any consequence derived from its use. Use it with care and
never trust the output without a careful meditation. 

This code has been developed by Andr\'es Asensio Ramos. It may not be copied
without acknowledging its origin. Use of this code is not restricted, provided
that acknowledgement is made in each publication. The bibliographic reference
to this code is Asensio Ramos, A. \& Trujillo Bueno, J., EAS Publications 
Series, 18, 2006, 25. Send bug
reports, comments and questions to A. Asensio Ramos (aasensio @ iac.es).


\newpage

%%%%%%%%%%%%%%%%%%%%%%%%%%%%%%%%%%%%%%%%%%%%%%%%%%
%%%%%%%%%%%%%%%%%%%%%%%%%%%%%%%%%%%%%%%%%%%%%%%%%%
\section{Introduction}

\G\ is a code for the solution of radiative transfer problems in
multi-level atomic systems in plane-parallel and spherical
geometries. The novel contribution of the code is that it
makes use of the very fast iterative methods based on Gauss-Seidel
and Successive Overrelaxation schemes to solve the problem.

The code is written in standard Fortran 90 and has been tested
with the Intel Fortran compiler and the GFortran compiler.

%%%%%%%%%%%%%%%%%%%%%%%%%%%%%%%%%%%%%%%%%%%%%%%%%%
%%%%%%%%%%%%%%%%%%%%%%%%%%%%%%%%%%%%%%%%%%%%%%%%%%
\section{Uncompressing and compiling \G}

The package comes in a single compressed file \texttt{gocciola.tar.gz}. After
unpacking with \texttt{tar zxvf gocciola.tar.gz}, the \G\ directory
will contain the master input file \texttt{config.dat} (see below) and the
following subdirectories:

\begin{itemize}
\item
{\tt Source} contains the Fortran 90 sources and a simple makefile.
\item
{\tt Atoms} contains molecular data files with energy levels,
A-coefficients and collision rates
\item
{\tt Atmos} contains several examples of model atmospheres, giving
the depth variation of several physical variables
\item
{\tt Results} contains the output of every run
\item
{\tt GUI} contains a preliminary version of a graphical user interface
to analyze the output of the code written in IDL
\end{itemize}

The code has been tested on several Linux platforms using the Intel Fortran
Compiler (ifort) and the GNU Fortran Compiler (\texttt{gfortran}). The
source code is in the \texttt{Source} directory. The compilation is performed
with the supplied \texttt{makefile}. It is quite simple and easy to modify, and
contains additional comments about compiling and pre-processing flags. The
default compiler is the free \texttt{gfortran} and you can use any other
compiler through the variable \texttt{COMPILER}. To compile the code, type:

\begin{footnotesize} \begin{verbatim}
       make clean
       make all
\end{verbatim} \end{footnotesize}
After compiling and linking, the executable is copied to the \G\
directory that contains the master input \texttt{config.dat}. Running the
program should produce output in the subdirectory {\tt Results}.


%%%%%%%%%%%%%%%%%%%%%%%%%%%%%%%%%%%%%%%%%%%%%%%%%%
%%%%%%%%%%%%%%%%%%%%%%%%%%%%%%%%%%%%%%%%%%%%%%%%%%
\section{Main configuration file}


%%%%%%%%%%%%%%%%%%%%%%%%%%%%%%%%%%%%%%%%%%%%%%%%%%%
%\section{Basic considerations}

The input file {\tt config.dat} has a strict format and lines cannot
be inserted without a crash of the program. For doing different calculations,
just use the provided file as a template. In order to show the structure of an input file, we take
the default file as example and consider all the input
parameters one by one.

\begin{footnotesize} \begin{verbatim}
# Verbose mode (0-> no, 1-> yes)
1
\end{verbatim} \end{footnotesize}
Sets the verbose mode on/off. If on, a lot of information is output to follow the iterative
convergence.

\begin{footnotesize} \begin{verbatim}
# (0 -> calculate pop, 1 -> write spectrum, 2 -> LTE spectrum, 3 -> previous run, 4 -> zeeman)
0
\end{verbatim} \end{footnotesize}
The standard option is to choose 0. In this case, it will solve the radiative
transfer problem and output all information in several files. Option 1 reads the
converged populations and writes the output spectrum. Option 2 calculates the spectrum
in local thermodynamical equilibrium. Option 3 starts the population from
a previous run.

\begin{footnotesize} \begin{verbatim}
# Interpolate the atmosphere (0-> no, n-> interpolate to n points)
0
\end{verbatim} \end{footnotesize}
In case you want the code to re-interpolate the model atmosphere to a certain
number of points. The typical value is 0 so that \G\ uses the model
provided by the user.

\begin{footnotesize} \begin{verbatim}
# Plane-parallel or spherical (SPHERICAL/PLANEP)
SPHERICAL
\end{verbatim} \end{footnotesize}
Select between spherical and plane-parallel calculations.

\begin{footnotesize} \begin{verbatim}
# Angle set for plane-parallel atmospheres (1/3)
3
\end{verbatim} \end{footnotesize}
Number of angles used for the angular quadratures for
calculating the mean intensity. The larger the number,
the better the calculations but also the slower.

\begin{footnotesize} \begin{verbatim}
# Initial population (0 -> LTE, otherwise -> LVG with indicated precision (if <0 then slow mode))
0
\end{verbatim} \end{footnotesize}
Initialize with local thermodynamical equilibrium populations using 0. If a positive
number is used, this indicates the precision for an LVG calculation used to
initialize (e.g., 10$^{-1}$).

\begin{footnotesize} \begin{verbatim}
# Characteristics going through the core
12
\end{verbatim} \end{footnotesize}
Number of rays in the spherical case that will cross the
inner core. The larger the number, the better the quality
of the result.

\begin{footnotesize} \begin{verbatim}
# Dust activation (3 -> Fabrice Herpin, 2 -> realistic, 1 -> simple, 0-> no)
0
\end{verbatim} \end{footnotesize}
In case you want to include dust opacity. Several options are possible and
all of them are still preliminary.

\begin{footnotesize} \begin{verbatim}
# Cosmic Microwave Background activation (1-> yes, 0 -> no)
1
\end{verbatim} \end{footnotesize}
Include the CMB radiation as a boundary condition.

\begin{footnotesize} \begin{verbatim}
# Background opacity activation (1 -> yes, 0 -> no)
0
\end{verbatim} \end{footnotesize}
For cases in which there is a source of background opacity, set it to 1.
The background opacity package include H$^-$, H, H$_2$ and Mg bound-free and free-free
contributions, Thomson scattering and Rayleigh scattering in H and H$_2$. This
package is more suitable for stellar atmospheres.

\begin{footnotesize} \begin{verbatim}
# Central source activation (3-> little source, 2 -> full source, 1 -> empty, 0-> absorbing (NO))
1
\end{verbatim} \end{footnotesize}
Is there a source inside the inner core? If no, select 1. If yes and it is filling
the core, select 2. If the source is smaller than the core, select 3. If the core
is just absorbing all radiation, select 0.

\begin{footnotesize} \begin{verbatim}
# Microwave background temperature (K)
2.726d0
\end{verbatim} \end{footnotesize}
Temperature of the CMB.

\begin{footnotesize} \begin{verbatim}
# Type of central source spectrum (BLACKB -> blackbody, GREYB -> grey body)
BLACKB
\end{verbatim} \end{footnotesize}
Two different central sources are available. Either a blackbody characterized
by a temperature or a grey body characterized by a temperature and some
spectral variation.

\begin{footnotesize} \begin{verbatim}
# Reference wavelength (microns), opacity at ref. wave. and spectral index
80.d0	2.5d0	1.d0
\end{verbatim} \end{footnotesize}
Data for the grey body.

\begin{footnotesize} \begin{verbatim}
# Central source temperature (K)
       1218.9840
\end{verbatim} \end{footnotesize}
Temperature of the central source.

\begin{footnotesize} \begin{verbatim}
# Atmosphere file
'Atmos/bernes.atmos'
\end{verbatim} \end{footnotesize}
File with the atmospheric model.

\begin{footnotesize} \begin{verbatim}
# Source distance (pc)
1.d3
\end{verbatim} \end{footnotesize}
Distance to the source.

\begin{footnotesize} \begin{verbatim}
# Source's angular diameter (arcsec)
26.73d0
\end{verbatim} \end{footnotesize}
Source angular size.

\begin{footnotesize} \begin{verbatim}
# Star radius (cm)
3.67653d11
\end{verbatim} \end{footnotesize}
Radius of the central source (just in case it is smaller than the core).

\begin{footnotesize} \begin{verbatim}
# Telescope's sigma (arcsec) and beam cut (arcsec) (if 0, no cut)
90.d0   0.d0
\end{verbatim} \end{footnotesize}
Size of the telescope beam in arcsec and size of the cutting in case
the beam is not complete.

\begin{footnotesize} \begin{verbatim}
# Observing instrument (LWS, SWS, ...)
LWS
\end{verbatim} \end{footnotesize}
Standard beams for some telescopes (not working at present).

\begin{footnotesize} \begin{verbatim}
# Save emerging spectrum (1 -> yes, 0-> no)
1
\end{verbatim} \end{footnotesize}
A flag to indicate whether to save the output spectrum or not.

\begin{footnotesize} \begin{verbatim}
# Save radiative rates (1 -> yes, 0-> no)
0
\end{verbatim} \end{footnotesize}
A flag to indicate whether to save the radiative rates or not.

\begin{footnotesize} \begin{verbatim}
# Include line overlapping (1 -> yes, 0-> no)
0
\end{verbatim} \end{footnotesize}
Include line overlapping (experimental) or not.

\begin{footnotesize} \begin{verbatim}
# Use collisional coefficients (1 -> yes, 0 -> no)
1
\end{verbatim} \end{footnotesize}
Include collisions or not.

\begin{footnotesize} \begin{verbatim}
# Output file
'Results/output.dat'
\end{verbatim} \end{footnotesize}
Output file with the summary of the results.

\begin{footnotesize} \begin{verbatim}
# Iteration file
'Results/iteration.dat'
\end{verbatim} \end{footnotesize}
Output file with the convergence history.

\begin{footnotesize} \begin{verbatim}
# Emerging spectrum output file
'Results/spectrum.dat'
\end{verbatim} \end{footnotesize}
Output file with the final intensity spectrum.

\begin{footnotesize} \begin{verbatim}
# Emerging flux output file
'Results/flux.dat'
\end{verbatim} \end{footnotesize}
Output file with the final flux spectrum.

\begin{footnotesize} \begin{verbatim}
# Final atmosphere output file
'Results/atmosphere.dat'
\end{verbatim} \end{footnotesize}
Output file with the atmospheric file and some additional information.

\begin{footnotesize} \begin{verbatim}
# Final background opacity output file
'Results/background.dat'
\end{verbatim} \end{footnotesize}
Output file with the background opacity data.

\begin{footnotesize} \begin{verbatim}
# Final radiative rates output file
'Results/radrates.dat'
\end{verbatim} \end{footnotesize}
Output file with the final radiative rates.

\begin{footnotesize} \begin{verbatim}
# Final collisional rates
'Results/collisrates.dat'
\end{verbatim} \end{footnotesize}
Output file with the final collisional rates.

\begin{footnotesize} \begin{verbatim}
# Maximum number of iterations
100
\end{verbatim} \end{footnotesize}
Maximum number of allowed iterations for convergence.

\begin{footnotesize} \begin{verbatim}
# Number of iterations before any acceleration
10
\end{verbatim} \end{footnotesize}
Iterations before acceleration. This is a typical value that works well
although sometimes it has to be increased if a degradation in the
convergence is detected.

\begin{footnotesize} \begin{verbatim}
# Number of steps before acceleration
5
\end{verbatim} \end{footnotesize}
Number of iterations between accelerations.

\begin{footnotesize} \begin{verbatim}
# Order of acceleration (Ng) Maximum = 4
2
\end{verbatim} \end{footnotesize}
Order of the Ng acceleration. This is a typical value.

\begin{footnotesize} \begin{verbatim}
# Type of acceleration (1 -> Ng,  otherwise -> No acceleration)
1
\end{verbatim} \end{footnotesize}
If you want to apply accelerations, use 1. If not, use 0.

\begin{footnotesize} \begin{verbatim}
# Minimum value to turn on exponential expansion
0.001d0
\end{verbatim} \end{footnotesize}
When the optical depth is very small, a series expansion is used for the
exponential function. This sets the limit for the optical depth to be considered small.
This is a good value.

\begin{footnotesize} \begin{verbatim}
# Maximum relative change of the solution
8.d-4
\end{verbatim} \end{footnotesize}
Stopping criterion. If the maximum relative error is smaller than this, stop.

\begin{footnotesize} \begin{verbatim}
# (NOT USED) Iterative improvement of the solution (0 -> no, otherwise -> number of iterations)
0
\end{verbatim} \end{footnotesize}
Not used.

\begin{footnotesize} \begin{verbatim}
# Linear system solver algorithm (0 -> LU solver, 1 -> SVD solver, 2 -> SLAP, 3 -> BiCGStab)
0
\end{verbatim} \end{footnotesize}
Use 0 or 1 (better 0). The rest of methods are at the moment no operative and
are only of interest for very large atomic/molecular models.

\begin{footnotesize} \begin{verbatim}
# Maximum relative change to turn on SNTB acceleration (not yet implemented)
1.d-1
\end{verbatim} \end{footnotesize}
Not implemented.

\begin{footnotesize} \begin{verbatim}
# Optical depth to consider as optically thin
1.d0
\end{verbatim} \end{footnotesize}
Not used right now.

\begin{footnotesize} \begin{verbatim}
# Number of points in the Gaussian quadrature for LVG
24
\end{verbatim} \end{footnotesize}
Number of points in the Gaussian quadrature for integrals in the LVG initialization.

\begin{footnotesize} \begin{verbatim}
# Atomic or molecular model
'Atoms/hco+.molecule'
\end{verbatim} \end{footnotesize}
File with the model atom/molecule.

\begin{footnotesize} \begin{verbatim}
# Use bound-free transitions (0 -> no, 1 -> yes)
0
\end{verbatim} \end{footnotesize}
Use bound-free transitions if present in the model. Not finished.

\begin{footnotesize} \begin{verbatim}
# Number of Doppler widths to sample the profile
8
\end{verbatim} \end{footnotesize}
Number of Doppler widths to use for the frequency integration in the
calculation of the mean intensity. This value should be more than enough
for moderate velocities.

\begin{footnotesize} \begin{verbatim}
# Number of frequency points to sample each Doppler width
4
\end{verbatim} \end{footnotesize}
Number of points in frequency per Doppler width. The larger, the better
the precision in the calculation of the mean intensity. This value gives good
results.

\begin{footnotesize} \begin{verbatim}
# Reference lambda for dust (microns)
180.d0
\end{verbatim} \end{footnotesize}
Some data for dust.

\begin{footnotesize} \begin{verbatim}
# Dust opacity exponent (lambda/lambda_ref)**(-exponent)
2.d0
\end{verbatim} \end{footnotesize}
Some data for dust.

\begin{footnotesize} \begin{verbatim}
# Helium abundance
0.2d0
\end{verbatim} \end{footnotesize}
Helium abundance in case collisions with He are used.

\begin{footnotesize} \begin{verbatim}
# Atomic hydrogen abundance (in case of molecular model)
0.0d0
\end{verbatim} \end{footnotesize}
Multiplication factor (if non-zero) for the H abundance.

\begin{footnotesize} \begin{verbatim}
# Error (0 -> convergence error, 1 -> true error (needs a previous run with 0)
0
\end{verbatim} \end{footnotesize}
Set it to zero.

\begin{footnotesize} \begin{verbatim}
# Iteration type (LAMBDA (1) /LAMBDAJAC (2) /JACOBI (3) /GS (4) /SOR (5) /SEMIGS (6) /SEMISOR (7) /NEWTON_CEP (8))
3
\end{verbatim} \end{footnotesize}
Iterative scheme. The most robusts are 2 and 3, which use the accelerated $\Lambda$ iteration.

\begin{footnotesize} \begin{verbatim}
# Omega (if SOR, then give value of omega in the following lines)
1.5d0
\end{verbatim} \end{footnotesize}
A parameter only used when converging with the SOR method.

\section{Atomic file}
In this section, we explain the basic format for an atomic/molecular
model. As an example, we choose the file \texttt{Atoms/hco+.molecule}.
\begin{footnotesize} \begin{verbatim}
# HCO+ molecular data file
# 
# 
# 
# 

# Atom or molecule (ATOM / MOLECULE)
MOLECULE
\end{verbatim} \end{footnotesize}
The first 5 lines can be filled with comments. The first
line selects between molecules and atoms because of the
slightly different treatment in the code.

\begin{footnotesize} \begin{verbatim}
# Name
HCO+
\end{verbatim} \end{footnotesize}
The name of the atom/molecule.

\begin{footnotesize} \begin{verbatim}
# Mass of element 
29.0
\end{verbatim} \end{footnotesize}
The mass of the atom/molecule in UMA.

\begin{footnotesize} \begin{verbatim}
# Number of ions
1
\end{verbatim} \end{footnotesize}
The number of ionization states. This is always 1 for molecules.

\begin{footnotesize} \begin{verbatim}
# Ionization levels
0.d0
\end{verbatim} \end{footnotesize}
The ionization levels in Hz.

\begin{footnotesize} \begin{verbatim}
# Number of levels
21
\end{verbatim} \end{footnotesize}
Number of energy levels.

\begin{footnotesize} \begin{verbatim}
# Number of transitions
210
\end{verbatim} \end{footnotesize}
Total number of transitions (radiative+collisional).

\begin{footnotesize} \begin{verbatim}
# Number of radiative transitions (bound-bound and bound-free)
20 0
\end{verbatim} \end{footnotesize}
Number of radiative transitions (bound-bound and bound-free).

\begin{footnotesize} \begin{verbatim}
# Number of active transitions
20
\end{verbatim} \end{footnotesize}
Number of active radiative transitions (transitions that are included
in the iterative process). Typically set this number equal to the number
of radiative transitions.

\begin{footnotesize} \begin{verbatim}
# Levels (E (cm^-1), g and ion)    	'------------------'
0.d0   1  1     'Level J=0         '
2.9750095  3  1 'Level J=1         '
8.9249613   5     1  'Level J=2  		 '
17.8497228   7     1  'Level J=3  		 '
29.7490978   9     1  'Level J=4  		 '
\end{verbatim} \end{footnotesize}
Information about the energy levels. The first number is the energy in cm$^{-1}$. The
second is the statistical weight. The third is the ionization state (always 1 for
molecules). The last column can be filled with comments to identify the level.

\begin{footnotesize} \begin{verbatim}
# Transitions (upper, lower, Aij for radiative ones)
2 1 4.217E-05
3 2 4.048E-04
4 3 1.464E-03
5 4 3.598E-03
6 5 7.186E-03
7 6 1.261E-02
8 7 2.024E-02
9 8 3.046E-02
10 9 4.365E-02
11  10 6.017E-02
12  11 8.042E-02
13  12  1.048E-01
14   13 1.336E-01
15   14 1.672E-01
16   15 2.061E-01
17   16 2.505E-01
18   17 3.009E-01
19   18 3.576E-01
20   19 4.210E-01
21   20 4.914E-01
\end{verbatim} \end{footnotesize}
Radiative transitions showing the index of the upper and lower level and the Einstein 
coefficient for spontaneous emission $A_\mathrm{ul}$ in s$^{-1}$.

\begin{footnotesize} \begin{verbatim}
# Collisional rates for all transitions
INTERPOL_H2
1 210
4
10.0
20.0
30.0
40.0

# Collisional data (first are radiative transitions)
 2   1   0.250000E-09 0.220000E-09 0.210000E-09 0.200000E-09
 3   2   0.450000E-09 0.400000E-09 0.370000E-09 0.350000E-09
 4   3   0.400000E-09 0.400000E-09 0.390000E-09 0.370000E-09
 5   4   0.380000E-09 0.380000E-09 0.370000E-09 0.380000E-09
 6   5   0.270000E-09 0.310000E-09 0.320000E-09 0.320000E-09
 7   6   0.280000E-09 0.280000E-09 0.280000E-09 0.280000E-09 
 8   7   0.280000E-09 0.280000E-09 0.280000E-09 0.280000E-09
 9   8   0.280000E-09 0.280000E-09 0.280000E-09 0.280000E-09
10   9   0.280000E-09 0.280000E-09 0.280000E-09 0.280000E-09  
11  10   0.280000E-09 0.280000E-09 0.280000E-09 0.280000E-09
12  11   0.280000E-09 0.280000E-09 0.280000E-09 0.280000E-09 
13  12   0.280000E-09 0.280000E-09 0.280000E-09 0.280000E-09
14  13   0.280000E-09 0.280000E-09 0.280000E-09 0.280000E-09
15  14   0.280000E-09 0.280000E-09 0.280000E-09 0.280000E-09
16  15   0.280000E-09 0.280000E-09 0.280000E-09 0.280000E-09
17  16   0.280000E-09 0.280000E-09 0.280000E-09 0.280000E-09
18  17   0.280000E-09 0.280000E-09 0.280000E-09 0.280000E-09
19  18   0.280000E-09 0.280000E-09 0.280000E-09 0.280000E-09
20  19   0.280000E-09 0.280000E-09 0.280000E-09 0.280000E-09
21  20   0.280000E-09 0.280000E-09 0.280000E-09 0.280000E-09   
 3   1   0.150000E-09 0.130000E-09 0.120000E-09 0.120000E-09
 4   1   0.830000E-10 0.730000E-10 0.630000E-10 0.660000E-10
 4   2   0.290000E-09 0.260000E-09 0.230000E-09 0.210000E-09
 5   1   0.830000E-10 0.740000E-10 0.670000E-10 0.670000E-10
 5   2   0.160000E-09 0.150000E-09 0.140000E-09 0.140000E-09
 5   3   0.290000E-09 0.290000E-09 0.280000E-09 0.270000E-09
...............
21  14   0.200000E-10 0.200000E-10 0.200000E-10 0.200000E-10
21  15   0.300000E-10 0.300000E-10 0.300000E-10 0.300000E-10
21  16   0.100000E-09 0.100000E-09 0.100000E-09 0.100000E-09
21  17   0.200000E-09 0.200000E-09 0.200000E-09 0.200000E-09
21  18   0.200000E-09 0.200000E-09 0.200000E-09 0.200000E-09
21  19   0.270000E-09 0.270000E-09 0.270000E-09 0.270000E-09

END
\end{verbatim} \end{footnotesize}
Collisional information. The first line indicates the type of
collisions. For molecules, use \texttt{INTERPOL\_H2}. The next
line indicates the initial and final collisional transition.
The next line indicates the number of temperatures for which
the collisional rates are given. Then, a large list comes with the
upper and lower level of each transition and the collisional
rate for each temperature. Note that the first transitions have
to coincide with those for radiative transitions.
 
\begin{footnotesize} \begin{verbatim}
# Partition function (if 0, then use partition function calculated from energy levels)
0
\end{verbatim} \end{footnotesize}
Give zero in case the partition function is calculated from the model. If you
want to use an already existing partition function, give the number of
coefficients and then, in another line, the coefficients of the interpolation.
Interpolation functions similar to those of Sauval \& Tatum (1984) are used.

\section{Atmospheric model}
In this section, we explain the basic format for an atmospheric model.
As an example, we choose the file \texttt{Atmos/hco+.atmos}.

\begin{footnotesize}
\begin{footnotesize} \begin{verbatim}
# TEST PROBLEM 1: HCO+ in inside-out collapse model: low abundance (all units are cgs)
#      r	(cm)		T(K)	  n(H2)		n(HCO+)		ne		v_mic(km/s)  T_dust   v_mac (km/s)
#
 50
  1.00065e+16   18.9000   254834.  1.00000e-09   0.0	0.15900       0.0 	  -0.7659920	0.0
  1.08209e+16   18.3635   235655.  1.00000e-09   0.0	0.15837       0.0 	  -0.7274062	0.0
  1.17017e+16   17.8423   217917.  1.00000e-09   0.0	0.15766       0.0 	  -0.6896906	0.0
  1.26541e+16   17.3358   201516.  1.00000e-09   0.0	0.15688       0.0 	  -0.6528440	0.0
  1.36840e+16   16.8437   186349.  1.00000e-09   0.0	0.15601       0.0 	  -0.6168676	0.0
  1.47977e+16   16.3656   172324.  1.00000e-09   0.0	0.15507       0.0 	  -0.5817597	0.0
  1.60021e+16   15.9011   159354.  1.00000e-09   0.0	0.15407       0.0 	  -0.5475215	0.0
  1.73046e+16   15.4497   147360.  1.00000e-09   0.0	0.15303       0.0 	  -0.5141520	0.0
  1.87130e+16   15.0112   136269.  1.00000e-09   0.0	0.15197       0.0 	  -0.4816522	0.0
\end{verbatim} \end{footnotesize}
\end{footnotesize}
The first three lines are comments and are freely available for the
user. The next line gives the number of shells of the model. Then,
for each shell, we have to give:
\begin{itemize}
 \item Radial position or height in cm. Note that for a plane-parallel atmosphere, the
absolute value of the position is irrelevant and only the difference between
two contiguous shells is important. For a spherical model, the first shell defines
the size of the inner core
\item Temperature in K
\item Hydrogen number density in cm$^{-3}$: H for atomic species and H$_2$ for molecular species
\item Atomic/molecular abundance. The number density of the species at each position is obtained
multiplying this number by the hydrogen abundance
\item Electron number density in cm$^{-3}$. It is only of importance for atomic calculations
\item Microturbulent velocity in km s$^{-1}$
\item Dust temperature in K in case dust is included
\item Macroscopic velocity in km s$^{-1}$
\item Gas pressure in dyn cm$^{-2}$ in case one wants to include background opacity in atomic calculations
\end{itemize}

\section{Output}
\G\ generates several files as output, all of them saved in the \texttt{Results} directory.
\begin{itemize}
\item \texttt{atmosphere.dat} contains information about the model atmosphere. The input
model atmosphere is copied, together with the computed impact parameters.
\item \texttt{output.dat} contains a summary of information, together with the
level populations, the departure coefficients, the mean intensity, the anisotropy and
information about each radiative transition, giving the variation with the optical
depth of the source function, the excitation temperature and the Planck function.
\item \texttt{spectrum.dat} gives the specific intensity for each frequency and angle.
\item \texttt{iteration.dat} gives the behavior of the convergence process.
\item \texttt{flux.dat} returns, as a first line, the number of transitions and the
number of frequency points for each line. Then, the flux in NLTE at each transition is listed
below. Then, the flux in LTE.
\item \texttt{collisrates.dat} lists the collision rates.
\end{itemize}


\section{Graphical User Interface}
Directory \texttt{GUI} contains a set of IDL programs for graphically
analyzing the results of a calculation. To use it, just run the
following in an IDL session:
\begin{verbatim}
IDL> @init
IDL> analyze
\end{verbatim}
The widgets are easy to use and should be easy to understand.

\newpage
\section*{Acknowledgements}
We would like to acknowledge R. Manso Sainz for useful discussions during the development
of the code and for suggesting the final name of the code. We thank
J. Cernicharo for his interest in the initial phases of the work. Finantial support
by the Spanish Ministry of Science and Innovation through projects AYAAYA2007-63881 and the European
Commission through the SOLAIRE network (MTRN-CT-2006-035484).

\bibliographystyle{apj}
\bibliography{/home/aasensio/Dropbox/biblio}


\end{document}
